\begin{frame}{Algoritmos - DFS e BFS}
    \begin{itemize}
        \item Algoritmos usados para buscar todos os vértices de um grafo
        \item Um vértice é dito descoberto quando ele é visitado pela busca
        \item Os vértices são representados como $v = \langle L_v, \pi_v, C_v \rangle$
        \item DFS
        \begin{itemize}
            \item [--] Sendo $v$ o vértice descoberto mais recentemente, uma busca é feita nas arestas partindo de $v$ por vértices não descobertos
            \item [--] Quando um novo vértice é descoberto, a busca passa a ser realizada nesse vértice
            \item [--] Quando um vértice sendo explorado não alcança diretamente nenhum vértice inexplorado, a busca é encerrada nesse vértice
        \end{itemize}
        \item BFS
        \begin{itemize}
            \item [--] Armazena os vértices recém descobertos em uma fila
            \item [--] Para cada vértice $u$ diretamente alcançável a partir de $v$, $u$ é inserido no final da fila
            \item [--] Após explorar cada aresta partindo de $v$, $v$ é removido da fila
        \end{itemize}
    \end{itemize}

    % \vspace{\baselineskip}

    % % São adicionados novas instâncias dos axiomas (K) e (Possibilidade), uma para cada \(\Box_i \text{ e }\Diamond_i\).
\end{frame}