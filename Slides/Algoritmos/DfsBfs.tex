\begin{frame}{Lógicas Multimodais}
    Extensão do conceito de lógica modais com apenas uma (ou um par de) modalidade(s) que contém diversas modalidades.
    A linguagem de uma lógica multimodal é o menor conjunto \(LM_n\) que respeita:

    \vspace{\baselineskip}

    \begin{enumerate}
        \item \(\top, \bot \in LM_n \)
        \item \(\mathbb{P} \subseteq LM_n\)
        \item \(\text{Se } \phi \in LM_n \text{, então } \circ \phi \in LM_n, \text{ sendo } \circ \in \{\Box_1, \dots, \Box_n, \Diamond_1, \dots, \Diamond_n, \neg\}\)
        \item \(\text{Se } \phi, \psi \in LM_n \text{, então } \phi \circ \psi \in LM_n, \text{ sendo } \circ \in \{\land, \lor, \to\}\)
    \end{enumerate}

    % \vspace{\baselineskip}

    % % São adicionados novas instâncias dos axiomas (K) e (Possibilidade), uma para cada \(\Box_i \text{ e }\Diamond_i\).
\end{frame}