\begin{frame}{Introdução}
    \begin{itemize}
        \item Lógica modal é o nome dado para uma família de lógicas não clássicas que lidam com \textit{modalidades};% - modos de interpretar proposições;
        \item Lógica não clássica é alguma lógica que quebra algum dos princípios da lógica clássica ou estende lógica clássica;
        \item Modalidade é um modo de interpretar uma fórmula~\cite{goldblatt1993mathematics};
        \begin{itemize}
            \item[--] Uma fórmula deve/pode ser verdadeira;
            \item[--] É obrigatório/aceitável que uma fórmula seja verdadeira
            \item[--] Conhecimento de um agente sobre uma fórmula
        \end{itemize}
    \end{itemize}
\end{frame}